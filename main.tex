\documentclass[11pt,a4paper,sans]{moderncv}
\usepackage{amsmath,amsthm,amssymb}
\usepackage{mathtext}
\usepackage[T1,T2A]{fontenc}
\usepackage[utf8]{inputenc}
\usepackage[english,bulgarian,ukrainian,russian]{babel}

\moderncvstyle{classic}                        % style options are 'casual' (default), 'classic', 'oldstyle' and 'banking'
\moderncvcolor{blue}                          % color options 'blue' (default), 'orange', 'green', 'red', 'purple', 'grey' and 'black'

\usepackage[scale=0.87]{geometry}
\setlength{\hintscolumnwidth}{3.3cm}           % if you want to change the width of the column with the dates

\firstname{Иван}
\familyname{Ершов}

\address{Москва}
\phone{89167199420}
\email{vanya.tugarin@yandex.ru}
\renewcommand\refname{Selected publications}
\nopagenumbers{}

\begin{document}
\makecvtitle
\section{Образование}
\cventry{2009 -- 2020}{Средняя Общеобразовательная Школа №3\newline с углубленным изучением английского языка г.о.Жуковский}{}{}{}{}
\cventry{2020 -- \\настоящее время}{НИУ ВШЭ (Москва)}{\newline Факультет компьтерных наук}{\newline Прикладная математика и информатика (GPA 8,71, Rating 11\%)}{}{}

\section{Образовательные курсы}
\cventry{2015 -- 2020}{курсы робототехники в ЦМИТ}{\newline (конструрование и программирование БПЛА)}{}{}{}
\cventry{2020 -- 2022}{Алгоритмы и Структуры данных, Архитектура компьютеров, \newline Линейная алгебра,  Математический анализ, Матричные вычисления, \newline Теория Вероятностей, Математическая Статистика}{}{}{}{}
\cventry{2022}{Школа Финтех при ФКН НИУ ВШЭ,\newline Курс по распознаванию лиц при магистратуре "Машинное обучение и высоконагруженные системы" ФКН НИУ ВШЭ}{}{}{}{}

\section{Навыки работы с компьютером}
\cvitem{Языки\\программирования}{С++ (STL), C,\newline Python (numpy, seaborn, pandas), SQL}
\cvitem{Ин}{git, LaTeX, Jupyter}
\section{Личные данные}
\cventry{Дата рождения}{5 Июля 2003}{}{Российская Федерация}{}{}
\cventry{Иностранные\\Языки}{Английский (B2+)}{}{}{}{}
\section{Немного о себе и проектах}
С дошкольного возраста изучал английский язык, а так как школа была с углубленным изучением английского языка , то он у меня на высоком уровне. Закончил школу с золотой медалью.
\\\\
В университете курс по Алгоритмам и Структурам Данных стал наиболее интересным (оценка 10), поэтому с легкостью могу найти практическое применение различным способам оптимизации в реальных задачах.
\\\\
Помимо точных наук я развиваю soft skills: помогаю однокурсникам разбираться в сложных темах, работаю учебным ассистентом по Математическому Анализу у первокурсников, обсуждаю с преподавателями возможность улучшить учебный процесс. 
\\\\
Последним проектом, над которым я работал, был «Анализ данных о движении человека». Было необходимо по имеющимся данным акселерометра носимого устройства (смартфон, электронные часы, фитнес-браслет) определить тип движения человека (ходьба, бег, велосипед, автомобиль и так далее)
\\\\
Сейчас веду работу над проектом «Система хранения и анализа данных носимых устройств», на котором по пульсу человеку нужно научиться выявлять различные состояния человека, таких как стресс или наличие заболевания. Впоследствии собранные данные могут использоваться для выдачи рекомендаций о приеме лекарств. 

	\end{document}
