\documentclass[11pt]{article}

\usepackage[top=1in, bottom=1in, left=0.5in, right=0.5in]{geometry}
\usepackage{amsfonts}
\usepackage[utf8]{inputenc}
\usepackage[russian]{babel}


\begin{document}
\begin{center}
\thispagestyle{empty}
\large \textbf{Ершов Иван Петрович \\}    
\normalsize vanya.ershoff@gmail.com $\diamond$ +7(916)-719-94-20 $\diamond$ github.com/tutugarin   \\
\hrulefill
\end{center}

\noindent \textbf{\underline{ОБРАЗОВАНИЕ}} \\
\par \textbf{НИУ "Высшая школа экономики"} \hfill Сентябрь 2020 - Настоящее время
\par Бакалаврская программа \textit{"Прикладная математика и информатика" } 
\begin{itemize}
\item[$\ast$] Изученные дисциплины: Алгоритмы и структуры данных, Архитектура компьютеров и операционные системы, Линейная алгебра, Математический анализ, Дискретная математика, Матричные вычисления, Теория вероятностей, Математическая статистика.

\item[$\ast$] GPA: 8.73/10
\end{itemize}
\par \textbf{Школа Финтех при ФКН НИУ ВШЭ}
\begin{itemize}
\item[$\ast$] Курс лекций по Финтех и применении Машинного обучения в нем от крупных компаний, таких как Сбербанк, OZON, Ренессанс Кредит, БАНК РОССИИ, MOEX, НСПК, банк "Открытие"
\end{itemize}

\par \textbf{Курс по распознаванию лиц при ФКН НИУ ВШЭ}
\begin{itemize}
\item[$\ast$] Работа с основными моделями ML. Линейная регрессия, логистическая регрессия, SVM. 
\end{itemize}

\noindent \textbf{\underline{ПРОЕКТЫ}} \\
\par \textbf{Анализ данных о движении человека}
\begin{itemize}
\item[$\ast$] Определение типа движения человека по данным с акселерометра смартфона
\item[$\ast$] используемые технологии: преобразование Фурье, Python 3
\end{itemize}

\par \textbf{Система хранения и анализа данных носимых устройств}
\begin{itemize}
\item[$\ast$] Командный проект (5 человек)
\item[$\ast$] Разработка Рекомендательной системы, Анализ сигнала ЭКГ
\item[$\ast$] \textbf{алгоритмы Машинного обучения:} Gradient Boosting, Random Forest, Neural Network
\end{itemize}

\noindent \textbf{\underline{ОПЫТ РАБОТЫ}} \\
\par \textbf{ООО "Яндекс Технологии"}\hfill 11.01.2022 - 13.06.2022
\begin{itemize}
    \item[$\ast$] \textbf{должность:} Стажер-Разработчик, Группа разработки Антиробота
    \item[$\ast$] Внедрение нового функционала. Разработка на C++ и Python. 
\end{itemize}

\noindent \textbf{\underline{ТЕХНИЧЕСКИЕ НАВЫКИ}}\\
 \par \textbf{Языки программирования:} \hfill Python 3.10 (numpy, pandas, sklearn, scipy),C++ (STL), SQL, C, Python 2\\
\par \textbf{Инструменты разработки:} \hfill CLion, PyCharm, VSCode, Git, Jupyter, \LaTeX{}, Shell script

\noindent \textbf{\underline{ДОПОЛНИТЕЛЬНАЯ ИНФОРМАЦИЯ}} \\
\par \textbf{Внеучебная деятельность:}\hfill Учебный ассистент, Куратор НИУ ВШЭ \\
\par \textbf{Владение языками:}\hfill Русский (родной), Английский (B2+)

\end{document}
